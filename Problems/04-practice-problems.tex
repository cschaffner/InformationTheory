\documentclass[a4paper,10pt,landscape,twocolumn]{scrartcl}

%% Settings
\newcommand\problemset{4}
\newcommand\deadline{Friday November 18th, 20:00h}
\newif\ifcomments
\commentsfalse % hide comments
%\commentstrue % show comments

% Packages
\usepackage[english]{exercises}
\usepackage{wasysym}
\usepackage{hyperref}
\hypersetup{colorlinks=true, urlcolor = blue, linkcolor = blue}

\begin{document}

\practiceproblems

{\sffamily\noindent
This week's exercises deal with AEP and encryption.  \practiceinstructions
}

\begin{exercise}[AEP and source coding]
A discrete memoryless source emits a sequence of of statistically independent binary digits with probabilities $P_X(1) = 0.005$ and $P_X(0) = 0.995$. The digits are taken 100 at a time and a binary codeword is provided for every sequence of 100 digits containing three or fewer 1's.
	\begin{subex}
	Assuming that all codewords are the same length, find the minimum length required to
provide codewords for all sequences with three or fewer 1's.
	\end{subex}
	\begin{subex}
		In the first homework problem set, you were asked to prove Markov's inequality.
		Use it to bound the probability of observing a source sequence for
		which no codeword has been assigned.
	\end{subex}
    \begin{subex}
    	In the first homework problem set, you were also asked to
        prove Chebyshev's inequality. Use it to bound the probability of observing a source sequence for
		which no codeword has been assigned.
	\begin{subex}
	Calculate the actual probability of observing a source sequence for which no codeword has
been assigned. Compare this number with the bounds computed
in part (b) and (c).
	\end{subex}
	\end{subex}
\end{exercise}



\begin{exercise}[The middle part of the entropy diagram]
Show that the value
\[
R(X;Y;Z) = I(X;Y) - I(X;Y|Z)
\]
is invariant under permutations of its arguments, using only the definitions and properties on the canvas page for \href{https://canvas.uva.nl/courses/2205/pages/definition-mutual-information?module_item_id=17286}{mutual information} and \href{https://canvas.uva.nl/courses/2205/pages/definition-conditional-mutual-information?module_item_id=26474}{conditional mutual information}.
\end{exercise}

\begin{exercise}[Entropy Inequalities]
For each statement below, specify a joint distribution $P_{XYZ}$ of random variables $X$, $Y$, and $Z$ ($P_{XY}$ of $X$ and $Y$ in (a)) such that the following inequalities hold.
\begin{subex}
There exists a $y$ such that $H(X|Y=y) > H(X)$.
\end{subex}
\begin{subex}
$I(X;Y) > I(X;Y|Z)$
\end{subex}
\begin{subex}
$I(X;Y) < I(X;Y|Z)$
\end{subex}
\end{exercise}

\begin{exercise}[Conditional mutual information]
Consider a sequence of $n$ binary random variables $X_1, X_2, ..., X_n$.
Each sequence with an even number of 1's has probability $2^{-(n-1)}$ and each sequence with an odd number
of 1's has probability 0. Find the mutual informations
\[
I(X_1;X_2), I(X_2;X_3|X_1), ..., I(X_{n-1};X_n|X_1, ..., X_{n-2}).
\]
\end{exercise}
\vspace{-0.5cm}

\begin{exercise}[Independence?]
	Let $(X_i,Y_i)$ be drawn i.i.d. according to $P_{XY}$. We compare the hypothesis that $X$ and $Y$ are independent to the hypothesis that they are dependent, by defining a random variable
	\[
	Z_n := \frac{P_{X^n}({X^n})P_{Y^n}(Y^n)}{P_{X^nY^n}(X^n,Y^n)}.
	\]
	What does $\frac{1}{n} \log Z_n$ converge to in probability? (\textbf{Hint:} look at the proof of the AEP for inspiration.)
\end{exercise}

\begin{exercise}[Bernoulli typical sets]
	Let $X_i$ be i.i.d. random variables, distributed according to a Bernoulli($p$) distribution: that is, $P_{X_i}(1) = p$ and $P_{X_i}(0) = 1-p$. Denote with $A_{\varepsilon}^{(n)}(p)$ the typical set for $X_1, ..., X_n$.
	
	\begin{subex}
		For arbitrary $\varepsilon > 0$, what is $A_{\varepsilon}^{(n)}(1/2)$?
	\end{subex}
	\begin{subex}
		Prove that for any $p < q \leq 1/2$, there exists
                an $\epsilon > 0$ such that for big enough $n$, it holds that $|A_{\varepsilon}^{(n)}(p)| < |A_{\varepsilon}^{(n)}(q)|$.
	\end{subex}
\begin{subex**}
	Prove that for any $p \neq q$ there exists an $\epsilon > 0$ such that for big enough $n$, it holds that $A_{\varepsilon}^{(n)}(p) \cap A_{\varepsilon}^{(n)}(q) = \emptyset$.
\end{subex**}
\end{exercise}






\end{document}