\documentclass[a4paper,10pt,landscape,twocolumn]{scrartcl}

%% Settings
\newcommand\problemset{4}
\newcommand\deadline{Friday November 18th, 20:00h}
\newif\ifcomments
\commentsfalse % hide comments
%\commentstrue % show comments

% Packages
\usepackage[english]{exercises}
\usepackage{wasysym}
\usepackage{hyperref}
\hypersetup{colorlinks=true, urlcolor = blue, linkcolor = blue}

\begin{document}

\practiceproblems

{\sffamily\noindent
This week's exercises deal with AEP and encryption.  \practiceinstructions
}

\begin{exercise}[AEP and source coding]
A discrete memoryless source emits a sequence of of statistically independent binary digits with probabilities $P_X(1) = 0.005$ and $P_X(0) = 0.995$. The digits are taken 100 at a time and a binary codeword is provided for every sequence of 100 digits containing three or fewer 1's.
	\begin{subex}
	Assuming that all codewords are the same length, find the minimum length required to
provide codewords for all sequences with three or fewer 1's.
	\end{subex}
	\begin{subex}
	Calculate the probability of observing a source sequence for which no codeword has
been assigned.
	\end{subex}
	\begin{subex}
	In the second homework problem set, you were asked to prove Chebyshev's inequality.
	Use it to bound the probability of observing a source sequence for
which no codeword has been assigned. Compare this bound with the actual probability computed
in part (b).
	\end{subex}
\end{exercise}



\begin{exercise}[Entropy diagram]
Show that the value
\[
R(X;Y;Z) = I(X;Y) - I(X;Y|Z)
\]
is invariant under permutations of its arguments.
\end{exercise}

\begin{exercise}[]
For each statement below, specify a joint distribution $P_{XYZ}$ of random variables $X$, $Y$, and $Z$ ($P_{XY}$ of $X$ and $Y$ in (a)) such that the following inequalities hold.
\begin{subex}
There exists a $y$ such that $H(X|Y=y) > H(X)$.
\end{subex}
\begin{subex}
$I(X;Y) > I(X;Y|Z)$
\end{subex}
\begin{subex}
$I(X;Y) < I(X;Y|Z)$
\end{subex}
\textbf{Note:} the distributions have to be different from the examples from the lecture or the lecture notes.
\end{exercise}

\begin{exercise}[Conditional mutual information]
Consider a sequence of $n$ binary random variables $X_1, X_2, ..., X_n$.
Each sequence with an even number of 1's has probability $2^{-(n-1)}$ and each sequence with an odd number
of 1's has probability 0. Find the mutual informations
\[
I(X_1;X_2), I(X_2;X_3|X_1), ..., I(X_{n-1};X_n|X_1, ..., X_{n-2}).
\]
\end{exercise}
\vspace{-0.5cm}






\end{document}