\documentclass[a4paper,10pt,landscape,twocolumn]{scrartcl}

%% Settings
\newcommand\problemset{2}
\newcommand\deadline{Wednesday, 14 November 2018, 12:00h}
\newif\ifcomments
\commentsfalse % hide comments
%\commentstrue % show comments

% Packages
\usepackage[english]{exercisespdf}
\usepackage{wasysym}
\usepackage{hyperref}
\hypersetup{colorlinks=true, urlcolor = blue, linkcolor = blue}

\usepackage{tikz}

\begin{document}

\homeworkproblems


\begin{exercise}[Huffman Coding]
	\begin{subex}[(4pt)]
    Consider the binary source $P_X$ with $P_X(0) = \frac{1}{8}$ and $P_X(1) = \frac{7}{8}$.
    
    Design a Huffman code for $P_X$. Describe, in order, which source symbols you combine, and list the final codebook. What is the average codeword length?
	\end{subex}
\begin{subex}[(4pt)]
	Design a Huffman code for blocks of $N=2$ bits drawn from the source $P_X$. Describe, in order, which source symbols you combine, and list the final codebook. What is the average codeword length?
\end{subex}
\begin{subex}[(4pt)]
	Design a Huffman code for blocks of $N=3$ bits drawn from the source $P_X$. Describe, in order, which source symbols you combine, and list the final codebook. What is the average codeword length?
\end{subex}
\begin{subex}[(4pt)]
	For the three codes you designed ($N=1,2,3$), divide the average codeword length by $N$, and compare these values to the optimal length, i.e., $H(X)$. What do you observe?
\end{subex}
	\begin{subex}[(1pt)]
	If you were asked to design a Huffman code for a block of $N = 100$ bits, what problem would you run into?
	\end{subex}
	\begin{subex}[(2pt)]
	Consider the random variable $Z$ with
	\begin{center}
	\begin{tabular}{c | c c c c c c}
	$z$ & 1 & 2 & 3 & 4 & 5 & 6\\
	\hline
	$P_Z(z)$ & 1/10 & 3/10 & 2/10 & 2/10 & 1/10 & 1/10\\
	\end{tabular}
	\end{center}
	Design a \emph{ternary} Huffman code for $Z$ (i.e. using an alphabet with three symbols).
	\end{subex}
\end{exercise}



\begin{exercise}[Inefficiency when using the wrong code]
\begin{subex}[(2pt)]
Given are two distributions $P_X$ and $Q_X$ for the same set $\mathcal{X} = \{\mathtt{a,b,c,d}\}$:
\[
\begin{array}{c | c c c c}
x & \mathtt{a}&\mathtt{b}&\mathtt{c}&\mathtt{d}\\\hline
P_X(x) & 1/4&1/4&1/4&1/4\\
Q_X(x) & 1/2&1/4&1/8&1/8\\
\end{array}
\]
Design an optimal prefix-free code for $Q_X(x)$.
\end{subex}
\begin{subex}[(4pt)]
What is the expected codeword length of the code you just designed? (Three decimals precision)
\end{subex}
\begin{subex}[(4pt)]
What is the expected codeword length if you use this code to encode the source $P_X$? (Three decimals precision)
\end{subex}
\begin{subex}[(4pt)]
Show that in general, when using optimal code for any source source $Q_X$ (which has lengths $\ell(x) = \lceil -\log Q_X(x) \rceil$ for all $x \in \mathcal{X}$) to encode another source $P_X$, this incurs a penalty of $D(P_X||Q_X)$ in the average description length.

More formally, prove that
\[
H(P_X) + D(P_X||Q_X) \leq \mathbb{E}_{P_X}[\ell(X)] \leq H(P_X) + D(P_X||Q_X) + 1
\]
for all $P_X$ and $Q_X$.
\end{subex}
\end{exercise}






\end{document}