\documentclass[a4paper,10pt,landscape,twocolumn]{scrartcl}

%% Settings
\newcommand\problemset{1}
\newcommand\deadline{na}
\newif\ifcomments
\commentsfalse % hide comments
%\commentstrue % show comments

% Packages
\usepackage[english]{exercises}
\usepackage{wasysym}
\usepackage{hyperref}
\hypersetup{colorlinks=true, urlcolor = blue, linkcolor = blue}

\begin{document}

\practiceproblems

{\sffamily\noindent
This week's exercises deal with bask probabilities.
You do not have to hand in these exercises, they are for practicing only. Problems marked with a $\bigstar$ are generally a bit harder. If you have questions about any of the exercises, please post them in the \href{https://canvas.uva.nl/courses/2205/discussion_topics/22948?module_item_id=130884}{discussion forum on Canvas}, and try to help each other. We will also keep an eye on the forum.
}

\enlargethispage{1cm}
\begin{exercise}[Two dice]
Consider an experiment where we throw two fair six-sided dice: a red one and a blue one.
	\begin{subex}
	What is the probability space $(\Omega,\mathcal{F},P)$ for this experiment? What would be the probability space if the dice were both red (i.e. indistinguishable)?
	\end{subex}
	\begin{subex}
	Let $X$ be the random variable that describes the sum of the two outcomes. Describe its range $\mathcal{X}$ and distribution $P_X$. What is $P_X(7) = P(X=7)$?
	\end{subex}
	\begin{subex}
	Let $Y$ be the random variable that describes the \emph{parity} of the sum, i.e. $\mathcal{Y} = \{\mathsf{even}, \mathsf{odd}\}$. What is $P_{X|Y}(7|\mathsf{odd})$? And $P_{X|Y}(7|\mathsf{even})$?
	\end{subex}
	\begin{subex}
	Verify that for an arbitrary random variable $X$, $(\mathcal{X},\mathcal{P}(\mathcal{X}),P_X)$ is a probability space.
	\end{subex}
\end{exercise}

\begin{exercise}[Inverse probabilities]
What is the probability that two (or more) students in this exercise class have the same birthday? (Assume everybody was born in the same year.)
\end{exercise}

\begin{exercise}[Events]
Let $\mathcal{A}, \mathcal{B}$ be events (subsets of some sample space $\Omega$). Prove the following identities:
	\begin{subex}
	$P[\overline{\mathcal{A}}] = 1 - P[\mathcal{A}]$
	\end{subex}
	
	\begin{subex}
	$P[\mathcal{A} \cup \mathcal{B}] = P[\mathcal{A}] + P[\mathcal{B}] - P[\mathcal{A},\mathcal{B}]$
	\end{subex}
	
	\begin{subex}
	$P[\mathcal{A}] = P[\mathcal{A},\mathcal{B}] + P[\mathcal{A},\overline{\mathcal{B}}]$
	\end{subex}
\end{exercise}

\begin{exercise}[Proof by induction]

	\begin{subex}
	Prove by induction on $n$ that for all $n \in \mathbb{N}_+$,
	\[
	\sum_{i=1}^n i = \frac{n(n+1)}{2}.
	\]
	\end{subex}
	
	\begin{subex}[Union bound]
	Prove the union bound for a finite number of events, which states that for arbitrary events $\mathcal{A}_1, \mathcal{A}_2, ..., \mathcal{A}_n$,
	\[
	P \left( \bigcup_{i=1}^n \mathcal{A}_i \right) \leq \sum_{i=1}^n P(\mathcal{A}_i).
	\]
	\end{subex}

	\begin{subex**}
	Can you find an exact formula for $P\left( \bigcup_{i=1}^n \mathcal{A}_i \right)$?
	\end{subex**}

\end{exercise}

\begin{exercise}[Probability urn]
There are 4 white and 5 black balls in a urn. In the first round, 2
balls are simultaneously drawn. If these two balls have different
colors, 3 white balls are added to the urn, if they have the same
color, 3 black balls are added. The two balls drawn in the first round
are discarded. In a second round, one more ball is drawn. What is the
probability that this last ball is white?
\end{exercise}

\begin{exercise}[Teams and slots]
	Suppose we have 10 teams labeled $T_1, ..., T_{10}$. Suppose they are ordered by placing their names in a hat and drawing the names out one at a time, successively placing them into slots numbered 1 to 10.
	
	\begin{subex}
		How many ways can it happen that all the odd numbered teams are in the odd numbered slots and all the even numbered teams are in the even numbered slots?
	\end{subex}
	\begin{subex}
		What is the probability that all even numbered teams end up in even numbered slots?
	\end{subex}
\end{exercise}


\enlargethispage{1cm}



\end{document}