\documentclass[a4paper,10pt,landscape,twocolumn]{scrartcl}

%% Settings
\newcommand\problemset{6}
\newcommand\deadline{Wed, , 20:00h}
\newif\ifcomments
\commentsfalse % hide comments
%\commentstrue % show comments

% Packages
\usepackage[english]{exercises}
\usepackage{wasysym}
\usepackage{hyperref}
\hypersetup{colorlinks=true, urlcolor = blue, linkcolor = blue}

\usepackage{tikz}

\begin{document}

\homeworkproblems

{\sffamily\noindent
Unless otherwise stated, you should provide exact answers rather than rounded numbers (e.g., $\log 3$ instead of 1.585) for non-programming exercises.
}

\newcommand{\ip}[2]{\langle #1, #2\rangle}
\begin{exercise}[Hadamard code (11pt)]
For two $k$-bit strings $x$ and $y$, the inner product is defined as
\[
\ip{x}{y} := \sum_{i=1}^k x_i\cdot y_i \ \ \ \ \ \ \ \ \mod 2.
\]
Note that the inner product of two strings is a single bit. The $k$-bit Hadamard code is defined such that the codeword of $x$ is the concatenation of all possible inner products with $x$, i.e.
\[
\mathtt{enc}(x) := \left(\ip{x}{y}\right)_{y \in \{0,1\}^k}.
\]
Here, the $y$ are ordered \href{https://en.wikipedia.org/wiki/Lexicographical_order}{lexicographically}.
	\begin{subex}[(1pt)]
	Find the codewords for 010 and 1101 using the 3-bit and 4-bit Hadamard codes respectively.
	\end{subex}
	\begin{subex}[(1pt)]
	Give the explicit generator matrix $G^T$ for $k = 3$.
	\end{subex}
	\begin{subex}[(2pt)]
	For a fixed non-zero $x \in \{0,1\}^k$, how many $y \in \{0,1\}^k$ are there such that $\ip{x}{y} = 1$?
	\end{subex}
	\begin{subex}[(1pt)]
	Use (c) to find the minimal distance of the $k$-bit Hadamard code.
	\end{subex}
There is some unnecessary redundancy in the Hadamard code. For example, the first bit of $\mathtt{enc}(x)$ is always 0, regardless of what $x$ is, because the inner product with the all-zero string is always 0. The \emph{punctured} $k$-bit Hadamard code attempts to resolve this. It considers only the inner products with strings $y$ such that $y_1 = 1$:
\[
\mathtt{enc}'(x) := \left(\ip{x}{y}\right)_{y \in \{1\} \times \{0,1\}^{k-1}}
\]
	\begin{subex}[(3pt)]
	Find the minimal distance of the punctured $k$-bit Hadamard code using a similar approach as in (c) and (d).
	\\\textbf{Hint:} there are four cases to consider, depending on the value of the first bit $x_1$ and of the ``tail" $x_2\cdots x_k$.
	\end{subex}
	\begin{subex}[(2pt)]
	Compute the rates of the $k$-bit Hadamard code and the punctured $k$-bit Hadamard code as a function of $k$. Which is better?
	\end{subex}
	\begin{subex}[(1pt)]
	What goes wrong if we try to puncture this code again? I.e. what breaks if we define the encoding function for a $k$-bit string as
	\[
	\mathtt{enc}''(x) := \left(\ip{x}{y}\right)_{y \in \{1\} \times \{1\} \times \{0,1\}^{k-2}}?
	\]
	\end{subex}
\end{exercise}

\begin{exercise}[Shannon capacity of the complete graph (6pt)]
A graph $G$ with $n$ vertices $V(G) = \{1,2,...,n\}$ is called complete if it has edges between any two vertices, i.e. $\forall i \neq j \ : \ (i,j) \in E(G)$.
	\begin{subex}[(2pt)]
	Compute $\alpha(K_n)$, the independence number of the complete graph of size $n$.
	\end{subex}
	\begin{subex}[(2pt)]
	Show that $K_n \boxtimes K_n = K_{n^2}$.
	\\\textbf{Hint:} the \LaTeX{} command for $\boxtimes$ is \texttt{$\backslash$boxtimes} (in the \texttt{amssymb} package). See also \href{http://detexify.kirelabs.org/classify.html}{Detexify}.
	\end{subex}
	\begin{subex}[(2pt)]
	Use (a) and (b) to prove that the Shannon capacity of $K_n$ is 0. Note that this result formally confirms the intuition that channels whose confusability graphs are complete are useless for zero-error communication, because all symbols can possibly be confused with each other.
	\end{subex}
\end{exercise}

\begin{exercise}[Disjoint graphs (5 pt)]
For two graphs $G$ and $H$, the graph $G + H$ is defined as the disjoint union of the two graphs. Formally, assuming without loss of generality that $V(G) \cap V(H) = \emptyset$, then $V(G + H) = V(G) \cup V(H)$ and $E(G+H) = E(G) \cup E(H)$. (You can think of $G + H$ as $G$ and $H$ ``next to each other''.)
	\begin{subex}[(2pt)]
	Prove that $\alpha(G+H) = \alpha(G) + \alpha(H)$.
	\end{subex}

	\begin{subex}[(3pt)]
	For any three graphs $G,H,L$, it holds that
	\[
	(G+H) \boxtimes L = (G \boxtimes L) + (H \boxtimes L)
	\]
	and also
	\[
	G \boxtimes (H + L) = (G \boxtimes H) + (G \boxtimes L).
	\]
	You can verify the above identities for yourself, but you do not have to hand in a proof. Use them to derive that for any natural number $k \in \mathbb{N}$, \[(G + G)^{\boxtimes k} = (G^{\boxtimes k})^{+2^k}.\]
	\end{subex}
\end{exercise}

\begin{exercise}[Same-parity channel (6pt)]
Let $\mathcal{X} = \mathcal{Y} = \{1,2,3,4,5,6\}$. In this exercise, you will compute the zero-error Shannon capacity of the noisy channel with transition probabilities $P_{Y|X}(y|x) = 1/3$ if and only if $x$ and $y$ have the same parity (i.e. $x \equiv y \mod 2$).
	\begin{subex}[(2pt)]
	Give the confusability graph $G$ of the noisy channel $P_{Y|X}$ described above.
	\end{subex}
	\begin{subex}[(4pt)]
	Compute the Shannon capacity of $G$.
	\\\textbf{Hint:} use several results from the previous two exercises.
	\end{subex}
\end{exercise}

\begin{exercise}[Toggle channel (6pt)]
Given two channels $(\mathcal{X}_1, P_{Y_1|X_1}, \mathcal{Y}_1)$ and $(\mathcal{X}_2, P_{Y_2|X_2}, \mathcal{Y}_2)$ with $\mathcal{X}_1 \cap \mathcal{X}_2 = \mathcal{Y}_1 \cap \mathcal{Y}_2 = \emptyset$, define the ``union channel'' $(\mathcal{X}, P_{Y|X},\mathcal{Y})$ by allowing the transmitter to choose between sending a signal through either channel 1 or channel 2 (but not both) each time. Let $C$ be the capacity of the new channel, and let $C_1$ and $C_2$ be the capacities of the ``component channels''.
\begin{subex}[(4pt)]
Prove that $2^C = 2^{C_1} + 2^{C_2}$. Solve any maximization problems analytically (i.e. by hand).
\\\textbf{Hint:} think of $P_X$ as a tree, where the first step decides whether to use the first or the second channel.
\end{subex}

\begin{subex}[(1pt)]
Use (a) to find the capacity of the following channel (in three decimals precision):
\begin{center}
\begin{tikzpicture}
\fill[black] (0,3) circle (1mm);
\fill[black] (0,2) circle (1mm);
\fill[black] (0,1) circle (1mm);
\fill[black] (2,0) circle (1mm);
\fill[black] (2,1) circle (1mm);
\fill[black] (2,2) circle (1mm);
\fill[black] (2,3) circle (1mm);

\draw (0,3) -- (2,3);
\draw (0,2) -- (2,2);
\draw (0,3) -- (2,2);
\draw (0,2) -- (2,3);
\draw (0,1) -- (2,1);
\draw (0,1) -- (2,0);

\node[anchor=south] at (1,3) {0.8};
\node[anchor=north] at (1,2) {0.8};
\node at (0.5,2.75) {0.2};
\node at (0.5,2.25) {0.2};

\node[anchor=south] at (1,1) {0.6};
\node at (1,0) {0.4};

\end{tikzpicture}
\end{center}
\end{subex}

\begin{subex}[(1pt)]
Use (a) to find the capacity of an arbitrary channel with capacity $C_1$ combined with an ideal channel with $k$ inputs.
\end{subex}
\end{exercise}





\end{document}