\documentclass[a4paper,10pt,landscape,twocolumn]{scrartcl}

%% Settings
\newcommand\problemset{6}
\newcommand\deadline{Friday November 18th, 20:00h}
\newif\ifcomments
\commentsfalse % hide comments
%\commentstrue % show comments

% Packages
\usepackage[english]{exercises}
\usepackage{wasysym}
\usepackage{hyperref}
\hypersetup{colorlinks=true, urlcolor = blue, linkcolor = blue}
\usepackage{tikz}

\begin{document}

\practiceproblems

{\sffamily\noindent
This week's exercises deal with channel capacities. You do not have to hand in these exercises, they are for practicing only. Problems marked with a $\bigstar$ are generally a bit harder. If you have questions about any of the exercises, please post them in the \href{https://www.moodle.ch/lms/mod/forum/view.php?id=1761}{discussion forum on Moodle}, and try to help each other. We will also keep an eye on the forum.
}

\begin{exercise}[Channel with memory]
[CT 7.36]
Consider the discrete memoryless channel with input alphabet $X_i \in \{1,-1\}$ and output $Y_i = Z_iX_i$.
\begin{subex}
What is the capacity of this channel when $\{Z_i\}$ is i.i.d. with $P[Z_i = 1] = P[Z_i = -1] = \frac{1}{2}$?
\end{subex}
\begin{subex}
Now consider the channel with memory. Before transmission begins, $Z$ is randomly chosen and fixed for all time. Thus, $Y_i = ZX_i$. What is the capacity of this channel when $P[Z = 1] = P[Z = -1] = \frac{1}{2}$?
\end{subex}
\end{exercise}

\begin{exercise}[Source and channel]
[CT 7.31]
We wich to encode a Bernoulli($\alpha$) process $V_1, V_2, ... V_n$ for transmission over a binary symmetric channel with crossover probability $p$, using the channel $n$ times.
\begin{center}
\begin{tikzpicture}
\node at (0,0) {$V^n$};
\node at (1.5,0) {$X^n(V^n)$};
\node at (3.15,0) {BSC$(p)$};
\node at (4.65,0) {$Y^n$};
\node at (6,0) {$\hat{V}^n$};
\draw[->,>=latex] (0.25,0) -- (0.75,0);
\draw[->,>=latex] (2.2,0) -- (2.6,0);
\draw[->,>=latex] (3.75,0) -- (4.25,0);
\draw[->,>=latex] (5,0) -- (5.5,0);
\end{tikzpicture}
\end{center}

Find conditions on $\alpha$ and $p$ so that the probability of error $P(\hat{V}^n \neq V^n)$ can be made to go to zero as $n \to \infty$.
\end{exercise}

\begin{exercise}[Encoder and decoder as part of the channel]
Consider a binary symmetric channel (BSC) $(\mathcal{X},P_{Y|X},\mathcal{Y})$ with crossover probability 0.1. A possible coding scheme for this channel with two codewords of length 3 is to encode message $w_1$ as 000 and $w_2$ as 111. The decoder uses majority vote. With this coding scheme, we can consider the combination of encoder, channel, and decoder as forming a new BSC $(\mathcal{X}',Q_{Y'|X'},\mathcal{Y}')$, with two inputs $w_1$ and $w_2$ and two outputs $w_1$ and $w_2$.
\begin{subex}
Draw this new channel and calculate its crossover probability.
\end{subex}
\begin{subex}
What is the capacity of this channel in bits per transmission of the original channel $P_{Y|X}$?
\end{subex}
\begin{subex}
What is the capacity of the original BSC $P_{Y|X}$ with crossover probability 0.1? Compare the two capacities.
\end{subex}
\begin{subex}
Prove the following general result: for any channel, considering the encoder, channel and decoder together (as a new channel from message $W$ to estimated messages $\hat{W}$) will not increase the capacity in bits per transmission of the original channel.
\end{subex}
\end{exercise}

\begin{exercise}[Practice!]
If you are done with all of the above problems, take the rest of the exercise session to practice older exercises from previous practice and/or homework problem sets.
\end{exercise}













\end{document}