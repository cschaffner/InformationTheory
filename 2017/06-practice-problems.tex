\documentclass[a4paper,10pt,landscape,twocolumn]{scrartcl}

%% Settings
\newcommand\problemset{6}
\newcommand\deadline{Friday November 18th, 20:00h}
\newif\ifcomments
\commentsfalse % hide comments
%\commentstrue % show comments

% Packages
\usepackage[english]{exercises}
%\usepackage{amsthm}
\usepackage{wasysym}
\usepackage{hyperref}
\hypersetup{colorlinks=true, urlcolor = blue, linkcolor = blue}
\usepackage{tikz}

\newtheorem{lemma}{Lemma}

\begin{document}

\practiceproblems

{\sffamily\noindent
This week's exercises deal with zero-error coding and channel capacity. You do not have to hand in these exercises, they are for practicing only. Problems marked with a $\bigstar$ are generally a bit harder. If you have questions about any of the exercises, please post them in the \href{https://www.moodle.ch/lms/mod/forum/view.php?id=2219}{discussion forum on Moodle}, and try to help each other. We will also keep an eye on the forum.
}

\begin{exercise}[Confusability graphs]
For each of the channels below, give the corresponding confusability graph.
	\begin{subex}
	$\mathcal{X} = \{1,2,3,4,5\}$, $\mathcal{Y} = \{a,b,c\}$, $P_{Y|X}(a|1) = P_{Y|X}(b|1) = P_{Y|X}(a|2) = P_{Y|X}(b|2) = \frac{1}{2}$, $P_{Y|X}(b|3) = \frac{1}{3}$, $P_{Y|X}(c|3) = \frac{2}{3}$, $P_{Y|X}(c|4) = P_{Y|X}(c|5) = 1$.
	\end{subex}
	\begin{subex}
	$\mathcal{X} = \{1,2,3,4,5\}$, $\mathcal{Y} = \{a,b,c,d\}$, $P_{Y|X}(a|2) = P_{Y|X}(b|2) = P_{Y|X}(c|2) = P_{Y|X}(a|4) = P_{Y|X}(c|4) = P_{Y|X}(d|4) = \frac{1}{3}$, $P_{Y|X}(b|3) = P_{Y|X}(c|3) = \frac{1}{2}$, $P_{Y|X}(a|1) = P_{Y|X}(d|5) = 1$.
	\end{subex}
\end{exercise}

\begin{exercise}[]
%(CT, Exercise 2.32)
We are given the following joint distribution of $X \in \{1,2,3\}$ and $Y \in \{a,b,c\}$:
\begin{align*}
P_{XY}(1,a) &= P_{XY}(2,b) = P_{XY}(3,c) = 1/6\\
P_{XY}(1,b) &= P_{XY}(1,c) = P_{XY}(2,a) = P_{XY}(2,c) = P_{XY}(3,a) = P_{XY}(3,b) = 1/12.
\end{align*}
Let $\hat{X}(Y)$ be an estimator for $X$ (based on $Y$), and let $p_{e} = P[\hat{X} \neq X]$.
	\begin{subex}
	Find an estimator $\hat{X}(Y)$ for which the probability of error $p_e$ is as small as possible.
	\end{subex}
	\begin{subex}
	Evaluate Fano's inequality for this problem, and compare.
	\end{subex}
\end{exercise}

\begin{exercise}[Symmetric Channels]
Consider the channel with transition matrix
\[
P_{Y|X} = \left[
\begin{array}{c c c}
0.3&0.2&0.5\\
0.5&0.3&0.2\\
0.2&0.5&0.3
\end{array}
\right].
\]
Recall that in a transition matrix, the entry in the $x$th row and the $y$th column denotes the conditional probability $P_{Y|X}(y|x)$ that $y$ is received when $x$ has been sent.

A channel is said to be \textbf{symmetric} if the rows of the channel transition matrix $P_{Y|X}$ are permutations of each other, and the columns are permutations of each other. A channel is said to be \textbf{weakly symmetric} if every row of the transition matrix is a permutation of every other row and all the column sums $\sum_x P_{Y|X}(y|x)$ are equal.

For instance, the channel $P_{Y|X}$ above is symmetric, and the channel
\[
Q_{Y|X} = \left[
\begin{array}{c c c}
\frac{1}{3}&\frac{1}{6}&\frac{1}{2}\\
\frac{1}{3}&\frac{1}{2}&\frac{1}{6}
\end{array}
\right]
\]
is weakly symmetric but not symmetric.
	\begin{subex}
	Find the optimal input distribution and channel capacity of $Q_{Y|X}$.
	\end{subex}
	\begin{subex}
	Give a general strategy how to compute the capacity of weakly symmetric channels. What is the optimal input distribution?
	\end{subex}
\end{exercise}

\begin{exercise}[Multiple Channel Uses]
Prove the lemma below stating that the capacity per transmission is not increased if we use a discrete memoryless channel many times. For inspiration, look again at the proof of the converse of Shannon's noisy-channel coding theorem.

\medskip
\noindent\textbf{Lemma 7.9.2 in [CT]}\  Let $X_1, X_2, \ldots X_n = X^n$ be $n$ random variables with arbitrary joint distribution $P_{X^n}$. Let $Y^n$ be the result of passing $X^n$ through a discrete memoryless channel of capacity $C$. Prove that for all $P_{X^n}$, it holds that $I(X^n; Y^n) \leq nC$.
\medskip

Does your proof also work in case of coding with feedback (i.e.\ $X_{i+1}$ is allowed to depend on $X^i$ and $Y^i$)? If not, point out the steps in your proof where you use that there is no feedback.
\end{exercise}

\begin{exercise}[Z-channel]
The following channel is known as the \emph{Z-channel}:
\begin{center}
\begin{tikzpicture}
\fill[black] (0,0) circle (1mm);
\fill[black] (0,2) circle (1mm);
\fill[black] (2,0) circle (1mm);
\fill[black] (2,2) circle (1mm);
\draw (0,2) -- (2,2);
\node[anchor=south] at (1,2) {1};
\draw (0,0) -- (2,0);
\node[anchor=north] at (1,0) {$1-f$};
\draw (0,0) -- (2,2);
\node[anchor=east] at (1,1) {$f$};

\node[anchor=east] at (0,2) {a};
\node[anchor=east] at (0,0) {b};
\node[anchor=west] at (2,2) {a};
\node[anchor=west] at (2,0) {b};
\end{tikzpicture}
\end{center}
Let $p \in [0,1]$ and write $P_X(a) = 1-p$, $P_X(b) = p$.

\begin{subex}
Express $I(X;Y)$ in terms of $p$ and $f$.
\end{subex}

\begin{subex}
Find the optimal input distribution and the channel capacity for $f = 0.3$.
\end{subex}
\begin{subex}
Do the same for $f = 0.15$ and $f = 0.01$. What do you observe?
\end{subex}
\end{exercise}













\end{document}